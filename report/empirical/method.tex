\chapter{Method}
The overall research method used in this study is action reasearch.
In this case a full iteration through the process was not conducted, time as usual, was not on our side. 
Nevertheless, some progress was made. This study is scheduled to be continued in a few months as of writing.
The work here will serve as the first two steps in the action research process, diagnoses and planning.
\section{Action Research}
Action research is based on an iterative process, plan, act and reflect.
Another way to describe this process is by:
\begin{description}
\item[Diagnosis:] Figure out what the problem is.
\item[Planning:] Figure out what to do about it.
\item[Intervention:] Execute the plan.
\item[Evaluation:] Comparing the results of actions taken with the proposed results in theory.
\item[Reflection:] Figure out if the goal was met, if any new knowledge was aquired and whether a new cycle is required.
\end{description}\cite{13}
The approach to set a diognosis was by conducting an exploratory case study. 
As with exploratory studies, the study should define the problem and uncover the theoretical topics which are relevant.
In short, it was a short-term, contemporary study. Meaning that is focused on getting an overview of the present situation.
\section{Selection of case}
The case was selected partly because it is a typical instance and partly of the unique opportunity.
Typical in the sense that the frames of the case was already set.
It was a health case that should be investigated within HISP. 
Within the HISP developer team I took part of a DHIS2 workshop in order to get familliar with the software.
While being included in meetings and email threads I've got in touch with HMIS team. 
They talked about two suitable cases were there could be useful to have some students working.
It was well suited for our purposes so we've decided to proceed.
\section{Data Collection}
As with all case studies and action research one has to generate some data to analyse.
In this case a practitioner-researcher approach was used\cite{13}. This is a way of combining working with a researchers hat.
Thing is, the case in question is already in my field of education. For working purposes I were an intern at the company hired to facilitate the HMIS in Rwanda.
This method of working has some drawbacks.
For obvious reasons the setting is somewhat artificial when people know that you are conducting a research while working.
This could be due to the fact that everythng they do and say could be recorded.

