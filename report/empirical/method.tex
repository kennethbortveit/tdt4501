\chapter{Method}
\section{Choice of Method}
In this study a combination of Case study and a practitioner-researcher would best describe the approach. 
To start with my research questions was not clearly defined, therefore an exploratory case study was needed. This also suits well because this study will make the foundation for a subsequent study that will take place in the next few months.
The practitioner-researcher method as a main data generation method was choosen much because of the circumstances. Since my objective was not clear until late in the research period, much time went on just making sense of what was really going on.
By actively participating in the day-to-day activities I could then get a feel for what problems that really stood in our way and what theme would best characterize the problem at hand. 
This approach was well suited since the primary work already is in my field of education. Cases vary in their approach to time, this particular case would best be described as a short-term, contemporary study. 
Case selection was very much based on a unique opportunity. After I decided that I would like to take a look at the DHIS as a case study, I pretty much had to wait for the oppertunity to participate. With some luck the oppertunity presented itself and the case was chosen.
The relationship to existion theory is to examine all the factors in the case and then see wich pre-existing theory or model best matches what was found in the case \cite[13]. 
\section{Data Collection}
As mentioned earlier, the main data collection is Observation. There is two main types of conduction observation, one is systematic observation and the other is Participant Observation. This case study focuses on the latter. There are also several way of conduction a particpant observation.
\begin{itemize}
\item Complete observer
\item Complete paricipant
\item Participant observer
\item Practitioner researcher
\end{itemize}
\cite{13}
I've chosen the Practitioner researcer approach. This is a way of observing while working at the task at hand. This method of working has some drawbacks. For obvious reasons the setting is somewhat artificial when people know that you are conducting a research.
Even though I tried to make my presence as natural as possible, I could sense that my co-workers act different around me than amongst themselves. As proposed by Briany J. Oates\cite{13}, this could be due to the fact that everythng they do could be recorded.
The main output of this kind of research is a theory of what is occuring.
\section{Reflection \& Data Analysis}
Looking back the time spent in the case, a total of 28 day, there is some things I would like to do differently in the future, given that the oppertunity. First of all, conducting a case study abroad takes alot of energy just adjusting to the new culture. 
After some time I found out that I was trying to isolate the case from everything else. 
It is very hard to isolate the case from all the other new impressions and maybe one shouldnt, but for effective reasoning and progress I found that this way would be best suited. This type of approach is likely to make in some way biased. 
The data presented in this report is highly subjective and should be treated as such. For starters it is a case study. The data produced are for the most part qualitative. This makes this kind of study very interpretive, but makes a very good foundation for what to look at next.
\subsection{7 Principles for Conduction and Evaluating}
