\documentclass[10pt, a4paper]{report}
\usepackage[utf8]{inputenc}
\usepackage{graphicx}
\usepackage{nomencl}
\usepackage{makeidx}

\makenomenclature
\title{{\Huge Interoperatability in Health Information Systems} \\ Report \\ Specialization Project \\ TDT4501}
\author{Kenneth Børtveit}
\begin{document}
\maketitle
\tableofcontents
\listoffigures
\listoftables
\printnomenclature
%Compile, then run this command to make the nomenclature
%makeindex  filename.nlo -s nomencl.ist -o filename.nls
%Then compile a couple of times.
\begin{abstract}

\end{abstract}
\chapter{Introduction}
\section{Research Questions}
What kind of tools and method of approach is necessary for optimeizing interoperability in developing countries?
\subsection{Interoperability}
Interoperatability between computing components may be generally defined as the ability to exchange information and mutually to use the information which has been exchanged \cite{6}. 

This report will focus on the challenges of interoperateability in the general electronic government  into a more specific case based scenario. The aim is to identify common challenges when trying to make the exchange of data between computer systems automated and accessable. The end result should be that the user of one system should be able to use data from external systems as if it were he's own.

\part{Litterature}
\chapter{Introduction}
Interoperatability between computing components may be generally defined as the ability to exchange information and mutually to use the information which has been exchanged \cite{6}. 

This report will focus on the challenges of interoperateability in the general electronic government  into a more specific case based scenario. The aim is to identify common challenges when trying to make the exchange of data between computer systems automated and accessable. The end result should be that the user of one system should be able to use data from external systems as if it were he's own.

\chapter{Interoperatability}
\chapter{Datawarehouse}
\chapter{Integration}
\chapter{Transition Strategy}
\section{System Migration \cite{2} \cite{8}}
\subsection{Initiation}
Who is the initiators and how does this impact the choice of system.
\subsection{Implementation}
\subsubsection{What characterizes a successful implementation}
\subsection{Cut-Over}
\subsubsection{Evolutionary vs. Revolutionary}
\subsection{Migration Methods}
\subsubsection{The Big Bang}
\subsubsection{Forward Migration}
\subsubsection{Backwords Migration}
\subsubsection{The Chicken Little Strategy}
\subsubsection{The Butterfly Methodology}
 
\part{Empirical}
\chapter{Context}
In the center of Africa we find the country known as Rwanda. A very small country, only \(26338 km^2\). This would be about 7\% of Norway. 
Their population is estimated to be around 12 million wich makes it about 420 people pr. square kilometer. 
Rwanda is made up of 5 provinces, east, west, north, south and Kigali. 
Each province is again divided into districts and there is a total of 30 districts. Under districts there is a total of 416 sectors\cite{1}.
\begin{figure}
\centering
\includegraphics[width=12cm]{empirical/images/rwanda_administrative_division}
\caption{Rwanda Administrative Division}
\end{figure}
It lies in the center of Africa with Uganda at the north, Tanzania to the east, Burundi at the south and the Democratic Republic of Congo in the west. Because of its location it works perfect as a gateway to all countries in Africa. 
And because of the stable environment comparing to the neighboring contries it is even more attractive for foreigners doing business in Africa making it the `Singapore of Africa'.

Rwanda has a goal of transforming to a knowledge based economy with Information and Communication Technology as their field of knowledge. This means basicly that they want to offer ICT services for other kind of resources. 
They want to be the regional center for the training of top quality ICT\nomenclature{ICT}{Information and Communication Technology} professionals.
This will hopefully in turn create wealth, jobs and entreprenaurs. From their perspective they have some competetive advantages in order to achieve this wich include:
\begin{itemize}
\item Cheap labor compared to other countries in the Region
\item Young and dynamic workforce (98\% of the population is under 50 years and 43\% is under 16 years)
\item Most favorable business environment in the Region (8th best place to do business in the world 2012)
\item Low levels of corruption - Zero tolerance (Transparency international Bribery index 2012 ranked Rwanda as least bribery prone in the EAC)
\item World class ICT infrastructure
\item Strong \& visionary leadership
\item Bi-lingual business environment (French and English)
\end{itemize}
\cite{2}

\section{Information Technology focus in Rwanda}

\begin{figure}
\centering
\includegraphics[width=12cm]{empirical/images/internet_penetration_2012}
\caption{Global Internet Penetration in 2012 \cite{3}}
\label{fig:global_internet_penetration_2012}
\end{figure}
\begin{figure}
\centering
\includegraphics[width=12cm]{empirical/images/internet_users_2012}
\caption{Global Internet Users in 2012 \cite{3}}
\label{fig:global_internet_users_2012}
\end{figure}
Rwanda has an internet penetration of 7\% in 2012. In Africa there is internet penetration is 15.6\% and for the world it is 34.3\% (See ~\ref{fig:global_internet_penetration_2012}).
The neighbouring countries, Uganda has 2.6\%, Tanzania 12.0\%, Burundi 1.7\% and the Democratic Republic of Congo with 1.2\%\cite{4}. Rwanda grew from 1\% to 7\% from 2000 to the end of 2011\cite{2}.
More interesting is the mobile broadband development in Rwanda. The subscriber base accounts for 48.1\% of the population and the network coverage accounts for 99.79\% of the country.
The government of Rwanda has made the decision to become an ICT hub in Africa. Therefore alot of resources and attention is focused on developing knowledge in the field of ICT. 
They are in 2012 ranked among the top 6 developing countries when it comes to dynamic performance in ICT development\cite{5}.
\subsection{The ICT park project}
As of january 2013 the Rwandian government is planning to set up an ICT park through the Rwanda Development Board.
This park will host technological training, industries research and development. The ICT park will support the growth of the following clusters:
\begin{itemize}
\item Energy
\item Internet, multimedia and mobile telecommunication
\item Knowledge
\item E-Government
\item Financial
\item ICT Service and export
\end{itemize}
\cite{2}

\section{Health Information Systems in Rwanda}
The government instance that has the responsibility to maintain and manage health information data is the Ministy of Health. Here there is a team that maintains the Health Management Information System.
The HMIS \nomenclature{HMIS}{Health Management Information System} is built on open source District Health Informaiton System 2. 
The health ministry has made some modifications so that there is in fact 4 instances of DHIS2\nomenclature{DHIS2}{District Health Information System 2} running for different purposes.
\begin{figure}
\centering
\includegraphics[width=12cm]{empirical/images/context.png}
\caption{Systems Currently in Use 2013}
\label{fig:systems_currently_in_use_2013}
\end{figure}
Besides the HMIS there are alot of systems that runs and has critical tasks that is not yet supported solely by DHIS2, see ~\ref{fig:systems_currently_in_use_2013}. These systems has varying tasks, but are all in some way related to HMIS.
Sharing data between these systems is crucial for maintaining an overview of the health status in all of Rwanda. 


\chapter{Method}

\chapter{Case}
\section{Health Information System Programme}
\subsection{About}

\subsection{History}

\section{District Health Information System}
\subsection{About}

\subsection{Modules}
\subsection{Usage}
\subsubsection{Use Case}

\section{Inforamation Systems in Rwanda}
\subsection{Current situation}

\subsection{Health Ministy Information System}
\subsection{DHIX}
\subsubsection{Apache Camel}
\subsection{Malaria Surveliance}
\subsubsection{Sentinel Surveliance}
\subsubsection{Active Surveliance}
\subsection{Voxivia}



\chapter{Discussion \& Conclusion}
Looking at the case result we see that the problem with the external systems is the formation of silos.
The external systems cannot share data with the HMIS. There might be reasons for why this is the case.
Like the owner of the external would like to have ownership of the data in order to be able to take care of their contract.
There might not be only positive effects of interoperability. Interoperability leads more efficient use of systems. 
And more efficency leads to redundency. Since old systems that are less efficient are no longer needed, the people facilitating these system are also no longer required to do so. Making people required to look for other endeavors. 
This might be a good argument for choosing an interoperability method that does not have external stakeholders in the development of a concrete solution.
The proposed access layer clearly advocates for a common data format solution, making Apache Camel a very contribution to this solution. The application would in some way synchronize databases with data. In that there is possibility of making changes to data with alot of dependencies. In this case the source system is not clearly defined, meaning that not all systems that will be affected are mapped. 
This clearly makes the redevelopment approach not suitable. The migration strategy would come in handy, but we are not really looking to move the system over to a new platform. I would be sufficient that we just provide some new functionality with the source system. Therefore a wrapper approach is the best choice, making no changes to the system already in place. Maintaining the system will then be a more complex job, but developing the access layer in a framework that is suitable for DHIS2 should be possible. The developer team is already planning to launch a DHIS2 appstore that would work perfectly as a framework for this kind of application. 

So by first being aware of the presence of system silos in a given context, we see that interoperability can be an issue. From here one should map the relevent systems that should be able to take advantage of an increase in interoperability. This would actually be another way of saying, identifying the stakeholders. One should choose a strategy based on the level of cooperation between stakeholders. In this case the organisations work pretty much as silos, so a common data format is the easiest to implement. From the interoperability strategy we get guidelines for what to develop. Then introducing the new system with the old requires a way of transitioning from the old way, and therefore choosing a transition strategy. In this case a wrapper that is built on top of the already existing system is best suited. In this case one would like to transition from propietary software to open source. For reasons discussed in section \ref{ict_in_dev}. As of now, Rwanda is charecterized as an developing country and would greatly benefit from an open source initiative.

\section{Acknowledgements}
Thank you all who have helped me during this research.



\chapter{Conclusion}
\begin{thebibliography}{99}
\bibitem{1}http://en.wikipedia.org/wiki/Rwanda , {\bfseries 2013}, {\itshape Wikipedia}
\bibitem{2}http://www.rdb.rw/rdb/ict.html, {\bfseries 2013}, {\itshape Rwanda Development Board}
\bibitem{3}Percentage of Individuals using the Internet 2000-2012,{\bfseries 2013}, {\itshape International Telecommunications Union (Geneva)}
\bibitem{4}http://www.internetworldstats.com/stats.htm, {\bfseries 2013}, {\itshape Internet World Stats}
\bibitem{5}http://www.newtimes.co.rw/news/index.php?a=62858\&i=15239, {\bfseries 2013}, {\itshape The New Times}
\bibitem{6}Luis Guijarro, {\bfseries 2006}, {\itshape Interoperability frameworks and enterprise architectures in e-government initiatives in Europe and the United States}
\end{thebibliography}
\appendix
%\input{appendix/journal/journal.tex}
\chapter{Malaria Reporting Form}
\label{chap:malaria_reporting_form}
\includegraphics[width=12cm]{appendix/pdf/malaria_reporting_2}


\end{document}