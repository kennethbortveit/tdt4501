\part{Introduction}
Interoperatability between computing components may be generally defined as the ability to exchange information and mutually to use the information which has been exchanged \cite{12}. 

This report will focus on the challenges of interoperateability in the general electronic government  into a more specific case based scenario. The aim is to identify common challenges when trying to make the exchange of data between computer systems automated and accessable. The end result should be that the user of one system should be able to use data from external systems as if it were he's own. The obvious answer to this problem would be to collect and store all data in one database, define a naming convention and then use this database as a central from which all systems can get reliable data and report data into. As it turns out, this is rarealy the case. For starters this requires a systems that meets the needs for all legacy and and future systems. Predicting the needs of systems not yet developed are a difficult task. Another basic issue is determine who is responsible for maintaining and develop such a system. 

