\part{Litterature}
\chapter{Introduction}
Interoperatability between computing components may be generally defined as the ability to exchange information and mutually to use the information which has been exchanged \cite{6}. 

This report will focus on the challenges of interoperateability in the general electronic government  into a more specific case based scenario. The aim is to identify common challenges when trying to make the exchange of data between computer systems automated and accessable. The end result should be that the user of one system should be able to use data from external systems as if it were he's own.

\chapter{Interoperatability}
\chapter{Datawarehouse}
\chapter{Integration}
\chapter{Transition Strategy}
\section{System Migration \cite{2} \cite{8}}
\subsection{Initiation}
Who is the initiators and how does this impact the choice of system.
\subsection{Implementation}
\subsubsection{What characterizes a successful implementation}
\subsection{Cut-Over}
\subsubsection{Evolutionary vs. Revolutionary}
\subsection{Migration Methods}
\subsubsection{The Big Bang}
\subsubsection{Forward Migration}
\subsubsection{Backwords Migration}
\subsubsection{The Chicken Little Strategy}
\subsubsection{The Butterfly Methodology}
