\chapter{Transition Strategy}
In order to make interoperability possible one as to transition from the old way of doing to the new.
And with a transition I talk about taking the system as it is and change it to something new. It's the process from old to new.
The process of transforming systems or system migration if you will.
\section{An overview}
Making a transition involves a switch from the old system to the new.
There is the source system, also referred to as the legacy system and the target system.
At one end of the specter we have the Big Bang strategy, were we taken on an revolutionary approach.
A complete new system is developed, supporting all the required functionality. 
Then one decideds a time when all of those involved switches to the new system. 
This way usually has a high risk of failure.
On the other end of the spectrum we have the evolutionary approach. Gradually one introduces new functionality, or the same with a new system, then after the legacy system is not used anymore, one turns off the switch.
\section{Planning and conducting a transition strategy}
There are some predifined methods for conducting a system migration which also could be usen in a transistion strategy plan.
Remembering that one moves from a source system into a target system.
As mentioned, solutions to transition problems could be characterized by how revolutionary it is.
The most revolutionary would be redevelopment, followed by migration, maintenance and finally wrapping.
One would choose the most appropriate strategy based on the level of risk. 
Like wrapping, one takes almost no risk, since it requires no real change to the system, but instead provides an updated interface for the source system.
Althogh this way is low risk, this could complicate things later on. Making use of wrapping not only slows down the system, but also makes maintenance more complicated.
The most appropriate use of wrapping is when one wants to make a new Graphical User Interface (GUI\nomenclature{GUI}{Graphical User Interface}).
Like when moving from a text based front-end to a graphical based front-end.
With redevelopment on one end and wrappng on the other, in the middle we have system migration. 
This technique allows for a smoother approach while being able to have control.
\section{Migration}
When redevelopment is to risky and wrapping is unsiutable, migration usually is the best way to go.
This allows for both systems to co-exist while making the transition from one to the other.
Migration usually involves moving an existing system to a new platform.
Before making the transition one has to decide on some basics. Like how one would like to migrate to the new system.
Much of the time is spent on testing the target system. Therefore it is good practice to not introduce new functionality while migrating to a new platform.
It also makes the testing easier since one could compare with the old system for output results.
New functionality should be introduced afters the old ones are supported.
\subsection{The cutover}
The cutover is the last step in the migration process. 
Here are three main approaches\cite{23}.
\begin{description}
\item[The cut and run]This is the most revolutionary way of migrating. It is much like redevelopment and seldom used alone. Once the target system is ready on turns of the source system and enable a new feature rich system.
\item[Phased interoperability]In this strategy incremantal steps towards the target system is used. Replacing functionality over time and slowly moving towards target system until all functionality is replaced. The last part of the cutover would be cut and run to some degree.
\item[Parallell operations]In this strategy both systems are running at the same time. Both source and target system is operational. The target system is continually tested and only when it's fully trusted, the source system is disabled.
\end{description}
The cut and run is very simple, but usaually involves high risk. Parallell operations usually become quite complex, but are fairly safe.
Phased interoperability is somewhere in the middle.
\subsection{Methods}
\subsubsection{The chicken little strategy}
\begin{enumerate}
\item Analyze the source system
\item Decompose the source system structure
\item Design the target interface
\item Design the target application
\item Design the target database
\item Install the target environment
\item Create and install necessary gateways
\item Migrate the source systems databases
\item Migrate the source systems applications
\item Migrate the source systems interfaces
\item Cut over to the target system
\end{enumerate}
The chicken little strategy is propesed by Michael Brodie and Michael Stonebreaker.
The method has severel steps towards migrating to the target system, making it a phased interoperability strategy.
It can be quite complex to handle source and target databases when using this approach.
\subsubsection{The butterfly}
This approach is a little different. One can here develop the different system in two seperate processes and making a kind of cut and run.
Before initation of the cutover, one freezes the source database and stores all datamanupilations in a temporary database. 
When the transfer is complete, another temporary database is used while transferring the temporary one.
Making the set of manipulatons smaller and smaller until transfer takes as little time that a cutover would not cause any problems.
The target system is then ready for use with all data ready to be used. 
