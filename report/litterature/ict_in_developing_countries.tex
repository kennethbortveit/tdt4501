\chapter{ICT in Developing Countries}
\label{ict_in_dev}
Lately there's been a growing focus on the global digital divide.
The main focus is the gap between developed and developing countries.
The global digital divide is an inequality in access, use and knowledge of information and communication technologies (ICT\nomenclature{ICT}{Inforfmation and Communication Technology}) from an international perspectice.

Analysis across countries has shown that education and income are the most likely to determine were on the ICT scale an idividual is\cite{21}.

\section{ICT for development}
The theory is that ICT's will further the development of society. 
ICT4D\nomenclature{ICT4D}{Information and Communication Technology for Development} (Information and Communication Technology for Development)  refers to using ICT's in socieconomic development, international development and human rights.
Since the gap is mainly between developed countries and developing countries, the focus for application is in developing countries.
An interesting estimate is that 40\% of the worlds population has on average 20\$ to use on ICT's pr. year \cite{22}.
Thus, ICT for development has to be low cost.
Open source projects are therefor a suitable candidate for these kinds of efforts.
In 2003 the World Summit on the Information Society held in Geneva, Switzerland, came up with an action plan on how ICT can support sustainable development.
They identified the following sectors to focus on.
\begin{itemize}
\item E-government
\item E-business
\item E-learning
\item E-health
\item E-employment
\item E-environment
\item E-agriculture
\item E-security
\end{itemize}\cite{22}
\section{Overcoming the digital divide}
An estimate shows that the borderline between ICT's as a necessity or luxury is around 10\$ pr. person pr. month.
Further, 40\% of the worlds population lives on less than 2\% a day. 20\% on less that 1\$ a day. 
So for the poorest 20\% this means that one would have to use one third of their available resources on ICT's.
The average in in the world would be about 3\%. 
The importance of low costs ICT is therefore obvious. 
Besides the economics there are two main barriers in order to overcome the digital divide.
Access and knowledge. The infrastructure has be in place in order for people make use of ICT's.
After the user has access, user has to know how to use the information systems in order to take advantage.
This has made people shift their focus not only on access and building infrastructure, but also to focus on teaching people how to use the technologies.
\section{Future problems}
Even in developed countries one has noticed a digital divide taking shape.
Applications has become so efficient that a users could basicly use ICT's without any previous knowledge. 
Creating a new digital divide between users and individuals able to develop applications.
Although this could works as a benefit, it is worth noting, so one can meet the future with some awareness.
The benefit might be that ICT''s will be of such quality that users can intergrate ICT's in their own field of knowledge without much effort. 
Seeing ICT's as a tool for accomplishing other tasks is important so that one does not loose sight of why the tool is made in the first place.
